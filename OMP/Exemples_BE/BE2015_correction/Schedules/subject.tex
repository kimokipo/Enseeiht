\documentclass{article}
\usepackage{marvosym}
\usepackage{dingbat}


\title{OpenMP parallel for scheduling}
\date{}

\begin{document}

\maketitle

The OpenMP parallel for worksharing construct supports different types
of scheduling. The objective of this exercise is to define a
scheduling strategy that is best suited for a specific code, propose a
parallelization and analyze the resulting behavior.

\section{Package content}
The \texttt{Schedules} directory contains a single file named
\texttt{schedules.c}. The relevant part of this file is represented by
the following two lines of code
\begin{verbatim}
  for (n=size; n>=100; n-=100){
    mat_mult(n, A+ptr[n], B+ptr[n], C+ptr[n]);
\end{verbatim}
At every iteration of this loop a matrix-matrix product of the form
$C=A\cdot B$ is performed, where $A$, $B$ and $C$ are dense matrices
of size \texttt{n}. Note that the $A$, $B$ and $C$ matrices are
different at every iteration and thus there are no loop-carried
dependencies. In the code, \texttt{size} is set to be equal to 1500.

The code can be compiled using the \texttt{make} command: just type
\texttt{make} inside the \texttt{Schedules} directory. This will
generate an executable \texttt{main} file that can be launched like
this
\begin{verbatim}
$ ./main
\end{verbatim}
Upon execution, the \texttt{main} program will print the execution
time for the loop above.

\section{Assignment}

\begin{itemize}
\item \smallpencil assuming the OpenMP parallel for construct has been
  used to parallelize the loop above, identify the scheduling type
  that is best suited for this code and explain your choice in the
  \texttt{responses.txt} file. Report in this file also the execution
  time of the code using 1, 2 and 4 threads with the scheduling type you
  have chosen and with the default, static scheduling. The observed
  results confirm what you expected?
\item {\huge \Keyboard} use the OpenMP parallel for construct to
  parallelize the loop above in the \texttt{schedules.c} file. Compile
  and run the program using 1, 2 and 4 threads, with both the scheduling
  type you have chosen and the default static scheduling; report the
  corresponding execution times in the \texttt{responses.txt} file as
  described above.
\end{itemize}



\end{document}

%%% Local Variables: 
%%% mode: latex
%%% TeX-master: t
%%% End: 
