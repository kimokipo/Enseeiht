\documentclass{article}


\setlength{\parindent}{0pt}
\date{07/12/2015}
\usepackage{marvosym}
\usepackage{dingbat}

\title{BE Syst\`emes Concurrents}

\begin{document}

\maketitle

This BE is made of five different exercises of increasing difficulty
for a total of 100 points:
\begin{description}
\item[Part 1] Schedules: an exercise on the usage of different
  scheduling types in the OpenMP parallel for worksharing
  construct. {\bf 15 points}.
\item[Part 2] Tree traversal: an exercise on the parallelization of a
  postordered binary tree traversal. {\bf 20 points}.
\item[Part 3] MergeSort: an exercise on the parallelization of the
  Merge Sort algorithm. {\bf 20 points}.
\item[Part 4] ConjugateGradient: an exercise on the parallelization
  of the basic operations used in the Conjugate Gradient iterative
  method. {\bf 20 points}.
\item[Part 5] PrefixScan: an exercise on the parallelization of the
  inclusive Prefix Scan operations. {\bf 25 points}.
\end{description}

For each exercise, detailed explanations and instructions are given in the
\texttt{subject.pdf} file inside the corresponding directory.

All the exercises include some coding tasks: these tasks consist in
writing, compiling and executing some OpenMP parallel code and are
identified by the keyboard symbol {\huge \Keyboard}. 

Some exercises also include more theoretical questions that are
identified by the pencil symbol \smallpencil. Please write the answer
to these questions in the \texttt{responses.txt} file in the topmost
directory of the BE package. {\bf Answers can be written in French}.

\vspace{1cm}

\textbf{General advice:}

\begin{itemize}
\item When implementing the parallelization, test your code on small
  data. When you're sure everything works fine, increase the size of
  data to evaluate performance.
\item All the proposed parallel solutions have to work with any
  (reasonable) number of threads. This means that the parallel code
  has to work also in the case where only one thread is used. Check
  your parallel code with one thread first; this case will be easier
  to debug in case of problems. Then test with more threads.
\item The amount of coding required in each exercise is relatively
  small. If you find yourself writing a lot of code, you're probably
  on the wrong track.
\end{itemize}


\vspace{1cm}


{\huge \bf Important}

Once you have finished execute the \texttt{pack.sh} script like this:
\begin{verbatim}
$ ./pack.sh
\end{verbatim}

This will generate a package containing the code you have developed
and the responses you have provided and automatically send it to the
supervisors.

{\bf Before leaving verify with the supervisor in your room that the
package has been received.}
\end{document}

%%% Local Variables: 
%%% mode: latex
%%% TeX-master: t
%%% End: 
